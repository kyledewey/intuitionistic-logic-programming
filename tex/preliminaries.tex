\section{Preliminaries}

It is assumed that the reader has some practical familiarity with classical logic programming, e.g., Prolog~\cite{roussel1975prolog, Warren:1977:PLI:872736.806939}.
This only means the capability to read and write simple logic programs; it is not necessary to understand the theoretical basis behind logic programming.
This will be provided as necessary.

First, a short dictionary of terms used throughout this document:

\begin{description}
  \item[LP] Short for ``logic programming'', which implicitly refers to classical logic programming (e.g., Prolog~\cite{roussel1975prolog, Warren:1977:PLI:872736.806939})

  \item[ILP] Short for ``intuitionistic logic programming''.
    Note that, in the literature, ``ILP'' usually refers to ``inductive logic porgramming'', which involves the automatic derivation of facts and rules about some input data.
    This is unrelated to intuitionistic logic programming.

  \item[ILLP] Short for ``intuitionistic linear logic programming''.
    While this document does not concern ILLP, it may be mentioned from time to time.

  \item[Ruleset] The facts and rules under which a given logical query is evaluated under, shown as \ruleset in mathematical notation.
\end{description}

