\section{Performing Hypothetical Reasoning in LP}
While ILP's big advantage is hypothetical reasoning, this is not to say that hypothetical reasoning is impossible to perform in LP.
It is, in fact, possible to perform hypothetical reasoning in LP, though it can quickly become unwieldy.
Because LP disallows modifying the rulebase, we must incorporate a sort of auxilliary rulebase as part of the query itself.
This auxilliary rulebase is passed around just like normal data, and can be manipulated as such.
To demonstrate how this works, consider the following code written in ILP style:

\begin{verbatim}
taken(alice, class1).
taken(bob,   class1).
taken(bob,   class2).

mayGraduate(S) :-
    taken(S, class1),
    taken(S, class2).

?- taken(alice, class2) => mayGraduate(S).
\end{verbatim}

The query in the above code is expected to nondeterministically bind \texttt{S} to \texttt{alice} and \texttt{bob}, representing the fact that if \texttt{alice} has \texttt{taken} \texttt{class2}, then both \texttt{alice} and \texttt{bob} \texttt{mayGraduate}.
In pure LP, we cannot perform hypothetical reasoning, and so the use of \texttt{=>} in the above code is forbidden.
Instead, we must encode \texttt{taken} directly in the queries, like so:

\begin{verbatim}
initialTaken([taken(alice, class1),
              taken(bob,   class1),
              taken(bob,   class2)]).

mayGraduate(S, Taken) :-
    member(taken(S, class1), Taken),
    member(taken(S, class2), Taken).

?- initialTaken(Initial),
   mayGraduate(S, [taken(alice, class2)|Initial]).
\end{verbatim}

In the above LP code, instead of accessing the rulebase directly to learn what classes we have taken, we instead record this information in a list holding a sequence of \texttt{taken} structures.
To add a fact to this list, we simply prepend an element to the front of the list using the \texttt{|} operator.
To determine if a given student has taken a class, we search the list for such a fact using \texttt{member}.

While the above code works, this is clearly cumbersome, and there are severe performance implications.
The following subsections describe these problems and possible solutions in detail.

\subsection{Passing Around Hypothetical Facts More Easily}
Forcing the user to explicitly pass around hypothetical facts is tedious and error-prone.
Every clause that transitively uses hypothetical facts must pass them around, which leads to code bloat.
A seemingly obvious solution to this problem from the functional programming community is to use a monad~\cite{Moggi:1989:CLM:77350.77353} to encapsulate the rulebase.
The monad itself can then take care of threading through the rulebase, effectively implicitly passing it around.
The problem with this approach is that monads assume the presence of higher-order functions, and pure LP lacks these features.
While it is possible to simulate these capabilities with extralogical metaprogramming constructs like \texttt{call}, this tends to be even more cumbersome and error-prone than explicitly passing hypothetical facts manually, not to mention major performance drawbacks of this approach.
A better idea here is to use definite clause grammars (DCG)~\cite{Abramson:1984:DCT:901674} to pass the facts around implicitly, which are already present in Prolog systems.
One can view DCGs as a highly specialized state monad implementation, which accomplishes its work via syntactic rewriting.
With DCGs, code clarity issues could be overcome, but performance issues remain.

\subsection{Performance Concerns of Hypothetical Facts}
The user pays a severe performance penalty for using data like a rulebase.
Consider a rulebase containing 10,000 \texttt{taken} relations, where \texttt{bob} is the student involved in just 2 of those relations.
If we specifically ask for relations involving \texttt{bob}, because of indexing optimizations~\cite{Ait-Kaci:1991:WAM:113900, AICPub641:1983}, the fact that the rulebase is so large tends to be irrelevant to the performance of this query.
That is, most engines need not scan over all 10,000 relations to find the two relations that involve \texttt{bob}.
However, with this ILP emulation, because the rulebase ends up being represented as a list, the same query must scan over all the 10,000 relations.
Forcing the user to index the list manually would solve the performance problem, but it passes what should be an engine concern on to the user.
As such, for performance reasons, it would be better to hide the hypothetical rulebase behind an API, which can handle maintaining an efficient form of the rulebase.
Additionally, it may be possible to piggyback some information off of the existing rulebase.
To demonstrate, consider the previous example that defined \texttt{initialTaken} as a list, which represented some base facts that were in the rulebase.
It is possible to shovel off initial ground facts to the rulebase without sacrificing anything, as long as we had a thin wrapper which would check for the presence of the fact either in the static rulebase or in the dynamic list of hypothetical facts.
